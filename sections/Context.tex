\section{Context \& Sensoren}

\subsection{Context}

\begin{frame}
    \frametitle{\insertsubsection} 
    \begin{block}{„Context“ \cite{context2015}}
        \vspace{0.3em}
        \small 
        ...eines Gerätes oder einer Anwendung:\\
        \vspace{1em}
        \normalsize
        Wissen über Umgebung -- intern und extern
    \end{block}
    \vspace{1em}
    Datenerhebung aus verschiedenen Quellen erlaubt Rückschluss auf Context
\end{frame}

\begin{frame}
    \frametitle{\insertsubsection} 
    Unterscheidung verschiedener Contexte \cite{context2015} 
    \begin{block}{Nutzungsdaten (Beispiele)}
        \begin{itemize}%[<+->]
            \setlength\itemsep{0.5em}
            \item Besuchte Webseiten
            \item Getätigte Anrufe
            \item Installierte Apps, Nutzungsstatistiken
        \end{itemize}
    \end{block}
    \vspace{0.5em}
    \begin{block}{Sensordaten (Beispiele)}
        \begin{itemize}%[<+->]
            \setlength\itemsep{0.5em}
            \item Helligkeit
            \item Beschleunigung
            \item GPS
        \end{itemize}
    \end{block}
\end{frame}

\subsection{Context Awareness}

\begin{frame}
    \frametitle{\insertsubsection \ \ \small \cite{context2015} \cite{orsini2016}} 
    „Context Awareness“
    \vspace{0.8em}
    \begin{itemize}
        \setlength\itemsep{0.7em}
        \item Gerät / Applikation erhebt und nutzt Context Daten
        \item App-Hersteller nutzen Context Daten
    \end{itemize}
    \vspace{2em}

    \begin{block}{„Context Awareness“ ist Nutzbar für:}
        \begin{itemize}%[<+->]
            \setlength\itemsep{0.7em}
            \item Userfeatures (zB. „Recommendations“, Lokalisierung, Health-Care, „Ambient Display“)
            \item App-Hersteller (zB. Verbesserungen der Produkte, gezielte Werbung)
        \end{itemize}
    \end{block}
\end{frame}

\subsection{Context Challenges}

\begin{frame}
    \frametitle{\insertsubsection} 
    \begin{block}{Challenges \cite{context2015} \cite{orsini2016}}
        \vspace{0.5em}
        Mobile-Charakteristika erschweren die Nutzung der Context Daten 
        \vspace{1em}
        \begin{itemize}%[<+->]
            \setlength\itemsep{1em}
            \item Heterogene Datenquellen (sehr viele Sensoren, zB. Auto)
            \item Limitierte Resourcen
            \begin{itemize}%[<+->]
                \setlength\itemsep{0.4em}
                \item Stromversorgung (Akku, Batterie) 
                \item Rechen-Resourcen
                \item Hoher Verbrauch durch sowohl Sensoren als auch Analyse der Daten!
            \end{itemize}
        \end{itemize}
    \end{block}
    \vspace{1em}
    \emph{„Datamining to the rescue“}
\end{frame}