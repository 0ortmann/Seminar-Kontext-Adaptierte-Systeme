\chapter{Context \& Sensoren}
\label{chap:context}

In diesem Kapitel wird knapp definiert was „Kontext“ ist und wie dieser ermittelt werden kann. Es werden einige anschauliche Beispiele genannt. Der Begriff von „Context Awareness“ wird geprägt. Schwierigkeiten bei der Kontexterhebung werden diskutiert.

\section{Kontext}

In diesem Aufsatz beschreibt der Begriff „Kontext“ das „Wissen über die interne oder externe Umgebung eines mobilen Gerätes oder einer Anwendung“. Damit richtet sich dieser Aufsatz nach der Definition aus \cite{context2015}. Mit geschickter Datenerhebung und -nutzung aus heterogenen Quellen lässt sich aufgrund von Messdaten auf den Kontext eines Gerätes oder einer App schließen \cite{orsini2016}.

Grob lassen sich nach \cite{context2015} zwei unterschiedliche Grundarten von erhobenen Daten klassifizieren. Zum einen \textit{Nutzungsdaten} zum anderen \textit{Sensordaten}. Nutzungsdaten beschreiben jene Daten, die softwareseitig erhoben werden. Das können Ausführungsdaten sein, die beispielsweise das Betriebssystem des mobilen Geräts misst oder direkt von einer Anwendung erfasste Benutzerdaten. Beispiele hierfür sind unteranderem besuchte Webseiten, getätigte Anrufe oder das Wissen über installierte und benutzte Apps sowie Nutzungsstatistiken. Sensordaten wiederum werden durch Sensoren auf dem mobilen Gerät erhoben. Moderne mobile Geräte verfügen über eine Vielzahl von Sensoren, etwa für die Messung der Helligkeit, Beschleunigung, Tempartur oder der geografischen Lage des Gerätes.

\section{Context Awareness \& Challanges}

Einzeln sind die gemessenen Daten nur begrenzt nützlich. Doch bei geschickter Kombination können die erhobenen Daten genutzt werden, um Rückschlüsse auf den Kontext des Gerätes oder einer App zu liefern. Die sinnvolle Nutzung dieser Daten beschreibt der Begriff „Context Awareness“ \cite{context2015, orsini2016}.

Ein Gerät oder eine App, die „context aware“ ist, kann zum Beispiel besser auf den Nutzer in einer bestimmten Situation abgestimmt werden. Die Umsetzung von mehr Userfeatures wird möglich durch den Einsatz von Kontextwissen. Beispiele hierfür sind etwa intelligente Kaufvorschläge („Recommendations“), Lokaliesrung, HealthCare oder eine sich automatisch anpassende Bildschirmhelligkeit.

Auch App-Hersteller profitieren von „Context Awareness“ von Geräten und Apps. Anwendungen können stark individualisiert und verbessert werden. Auch gezieltere Werbung wird möglich, die bestimmte auf bestimmte Nutzergruppen zugenschnitten ist.

Mit den entstehden Möglichkeiten der Datenerhebung sind gleichsam diverse Herausforderungen verbunden \cite{context2015, orsini2016}. Die Heterogenität der Daten erschwert die Nutzung, denn verschiedene Formate und Datenmengen müssen beabeitet werden können. Auf mobilen Geräten stehen allerdings nur begrenzt Resourcen zur Verfügung, insbesondere Akkuleistung und Rechenkapazität.