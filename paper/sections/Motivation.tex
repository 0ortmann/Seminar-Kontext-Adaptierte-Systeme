\chapter{Einführung}

Durch mobile Geräte werden zunehmend mehr Daten erhoben. Diese Daten sind unterschiedlicher Herkunft, sie werden etwa von Sensoren aufgenommen oder sind statistisch erhobene Nutzungsdaten von Apps. Durch eine geschickte Kombination dieser Daten lassen sich sogenannte „Kontexte“ ableiten, in denen das Gerät oder bestimmte Anwendungen genutzt werden.

Diese Aufsatz stellt zunächst verschiedene Formen von „Kontext“ in Kapitel \ref{chap:context} vor. Schwierigkeiten bei der Kontexterhebung werden ausgeleuchtet. Anschließend folgt in Kapitel \ref{chap:datamining} eine detaillierte Vorstellung einiger gängiger Datamining Verfahren und Algorithmen. Es wird diskutiert, inwiefern diese Verfahren für den mobilen Bereich geeignet sind. Herausforderungen für Datamining mit und auf mobilen Geräten werden dargestellt. In Kapitel \ref{chap:realworld} wird zunächst ein beispielhafter Überblick über wirtschaftliche Interessen an mobilem Datamining geboten. Es folgt eine kurze Analyse einer bekannten App in Bezug auf deren Einsatz von Datamining. Abschließend folgt eine Zusammenfassung. \\

Ziel dieses Aufsatzes ist dem Leser einen Überblick über Datamining im mobilen Bereich zu vermitteln. Diverse Algorithmen werden knapp theoretisch erläutert. Sowohl wissenschaftliche als auch wirtschaftliche Motive, Datamining derart zu betreiben, sollen vermittelt werden.

\nocite{*}