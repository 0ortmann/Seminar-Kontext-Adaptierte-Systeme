% ---------------------------------------------------------------------
% Das Dokument kompiliert mit pdflatex und ist auf Basis 
% von Koma-Script entstanden. 
%
% Autor des Templates (für Anmerkungen): 
% Michael von Riegen, riegen@informatik.uni-hamburg.de
%
% Einzelne Code-Teile für das Titelblatt sind aus dem Template 
% von Benjamin Kirchheim entnommen.
% ---------------------------------------------------------------------
\documentclass[11pt,bibtotoc,DIV15,BCOR20mm,tablecaptionabove,openany]{scrbook}

% packages
\usepackage[ngerman]{babel}
\usepackage[utf8]{inputenc}
\usepackage[T1]{fontenc}

\usepackage[pdftex]{color} 
\usepackage{setspace}
\usepackage[pdftex]{graphicx}  
\usepackage[final]{listings}  
\usepackage{ae,aecompl}  
\usepackage[normalem]{ulem}
\usepackage{amsfonts} % ok-zeichen usw
\usepackage{scrpage2}

% schrift auf palatino umstellen
\usepackage{palatino}
\setkomafont{sectioning}{\normalcolor\bfseries}

% pdfbookmarks 
% (generiert für PDF auch gleich 
% ein Inhaltsverzeichnis und Links)
\usepackage[
    pdfstartview=FitH,   
    pdffitwindow=true,
    colorlinks,
    linkcolor=black,
    anchorcolor=black,
    citecolor=black,
    urlcolor=black
]{hyperref}

% Zeilenabstand
\setstretch{1.24}   

% Pagestyle definieren (nach Martins Template)
\defpagestyle{diplHeadings}
{
    % obere Linie
    (0pt,0pt)
    % linke Seite
    {\upshape \rlap{\pagemark} \hfill \headmark} 
    % rechte Seite
    {\upshape \headmark \hfill \llap{\pagemark}} 
    % falls Layout "one page", aber das haben wir ja nicht
    {}
    % untere Line
    (\textwidth,1pt)
}
{
    % obere Linie
    (\textwidth,1pt)
    % linke Seite
    {}
    % rechte Seite
    {}
    % falls Layout "one page", aber das haben wir ja nicht
    {}
    % untere Linie
    (0pt,0pt)
}  
% Pagestyle auch für Chapter-Anfang einrichten
\renewcommand*{\chapterpagestyle}{diplHeadings}
\renewcommand*{\chapterheadstartvskip}{\vspace*{-\topskip}}
\automark[section]{chapter}

% Ränder  
\setlength{\textwidth}{15cm}        % Textbreite
\setlength{\textheight}{24cm}       % Texthöhe
\setlength{\topmargin}{-12mm}       % oberer Rand

% PDF Dokumentinformationen
\hypersetup {
    pdftitle     = {Titel},
    pdfsubject   = {Universität Hamburg / FB18 / VSIS / Art des Artikels},
    pdfauthor    = {Vorname Name},
    pdfkeywords  = {Keywords},
    plainpages   = false, pdfpagelabels
}

% Listingsformatierung (Quelltexte)
\definecolor{lightgrey}{rgb}{0.90,0.90,0.90}
\lstloadlanguages{XML}
\lstset {
    tabsize=2,
    escapeinside={(*@}{@*)},
    captionpos=t,
    framerule=0pt,
    backgroundcolor=\color{lightgrey},
    basicstyle=\small\ttfamily,
    keywordstyle=\small\bfseries,
    numbers=left,
    fontadjust
}

\begin{document}

% TITELSEITE
\begin{titlepage}

	% Fehler "destination with the same identifier" unterdrücken...
  \setcounter{page}{-1}

	% Titelseite
	\begin{figure}[h]
		\begin{minipage}[b]{25mm}
			\includegraphics[width=25mm,clip]{img/vsislogo}
		\end{minipage}
		\begin{minipage}[b]{2mm}
			\includegraphics[width=1mm,height=24mm]{img/greypixel}
		\end{minipage}
		\begin{minipage}[b]{12 cm}
			{\sffamily
				{\Large Universität Hamburg } \\
				Fakultät für Mathematik, \\
				Informatik und Naturwissenschaften \\
				\\
			}
		\end{minipage}
	\end{figure}

	\vfill
	
	\begin{center}
		% Seminararbeit 
		\noindent { \huge
			Hausarbeit im Seminar: \\
			{Mobile kontextadaptive Dienste und Systeme}\\
		}
		\vspace{14mm}
		% Titel
		\noindent \textbf{\huge
		  Data Mining \& Analyse von Sensordaten\\
		}
		\vspace{60mm}	
	\end{center}
	
	\vfill
	
	\noindent \textbf{Felix Ortmann} \\
	\noindent \rule{\textwidth}{0.4mm} 
	\noindent{\textrm{0ortmann@informatik.uni-hamburg.de}} \\
	\noindent{\textrm{Studiengang MSc. Informatik}} \\
	\noindent{\textrm{Matr.-Nr. 6383452}} \\
	\noindent{\textrm{Fachsemester 9}} \\
	\begin{tabbing}
	\hspace{8em} \=  \kill
	Betreuer: \> Prof. Dr. Winfried Lamersdorf \\
						\> Gabriel Orsini \\
						\> Wolf Posdorfer \\

	\end{tabbing}
	
	\setcounter{page}{0}
~

\end{titlepage}



% pagestyle auf seminararbeit umschalten
\pagestyle{diplHeadings}

% VERZEICHNISSE (Inhaltsverzeichnis, Abkürzungen)
% --> Seitennummerierung römisch
\pagenumbering{Roman}
\setcounter{page}{1}
% Inhaltsverzeichnis
\tableofcontents
\cleardoublepage

% KAPITEL
% --> Seitennummerierung wieder arabisch
\pagenumbering{arabic}
\setcounter{page}{1}

\section{Einführung}


\subsection{Motivation}

\begin{frame}
    \frametitle{\insertsubsection} 
    \begin{itemize}
        \setlength\itemsep{1em}
        \item Durch mobile Geräte werden zunehmend mehr Daten erhoben
        \item „Context“ Wissen
        \item Unterschiedlichste Nutzungsszenarien
        \item Nutzbar für wen, wozu?
        \item Besondere Verfahren notwendig, um nutzbare Informationen aus Datenflut zu extrahieren
    \end{itemize}
\end{frame}


\subsection{Zielsetzung}

\begin{frame}
    \frametitle{\insertsubsection} 
    \begin{itemize}
        \setlength\itemsep{1em}
        \item Datamining und Machinelearning mit/auf mobilen Geräten
        \item Hearausstellung diverser Problematiken bezogen auf den mobilen Bereich
        \item Datamining, Techniken und Algorithmen
        \item Real-World Beispiele
    \end{itemize}
\end{frame}

\section{Context \& Sensoren}

\subsection{Context}

\begin{frame}
    \frametitle{\insertsubsection} 
    \begin{block}{„Context“ \cite{context2015}}
        \vspace{0.3em}
        \small 
        ...eines Gerätes oder einer Anwendung:\\
        \vspace{1em}
        \normalsize
        Wissen über Umgebung -- intern und extern
    \end{block}
    \vspace{1em}
    Datenerhebung aus verschiedenen Quellen erlaubt Rückschluss auf Context
\end{frame}

\begin{frame}
    \frametitle{\insertsubsection} 
    Unterscheidung verschiedener Contexte \cite{context2015} 
    \begin{block}{Nutzungsdaten (Beispiele)}
        \begin{itemize}%[<+->]
            \setlength\itemsep{0.5em}
            \item Besuchte Webseiten
            \item Getätigte Anrufe
            \item Installierte Apps, Nutzungsstatistiken
        \end{itemize}
    \end{block}
    \vspace{0.5em}
    \begin{block}{Sensordaten (Beispiele)}
        \begin{itemize}%[<+->]
            \setlength\itemsep{0.5em}
            \item Helligkeit
            \item Beschleunigung
            \item GPS
        \end{itemize}
    \end{block}
\end{frame}

\subsection{Context Awareness}

\begin{frame}
    \frametitle{\insertsubsection \ \ \small \cite{context2015} \cite{orsini2016}} 
    „Context Awareness“
    \vspace{0.8em}
    \begin{itemize}
        \setlength\itemsep{0.7em}
        \item Gerät / Applikation erhebt und nutzt Context Daten
        \item App-Hersteller nutzen Context Daten
    \end{itemize}
    \vspace{2em}

    \begin{block}{„Context Awareness“ ist Nutzbar für:}
        \begin{itemize}%[<+->]
            \setlength\itemsep{0.7em}
            \item Userfeatures (zB. „Recommendations“, Lokalisierung, Health-Care, „Ambient Display“)
            \item App-Hersteller (zB. Verbesserungen der Produkte, gezielte Werbung)
        \end{itemize}
    \end{block}
\end{frame}

\subsection{Context Challenges}

\begin{frame}
    \frametitle{\insertsubsection} 
    \begin{block}{Challenges \cite{context2015} \cite{orsini2016}}
        \vspace{0.5em}
        Mobile-Charakteristika erschweren die Nutzung der Context Daten 
        \vspace{1em}
        \begin{itemize}%[<+->]
            \setlength\itemsep{1em}
            \item Heterogene Datenquellen (sehr viele Sensoren, zB. Auto)
            \item Limitierte Resourcen
            \begin{itemize}%[<+->]
                \setlength\itemsep{0.4em}
                \item Stromversorgung (Akku, Batterie) 
                \item Rechen-Resourcen
                \item Hoher Verbrauch durch sowohl Sensoren als auch Analyse der Daten!
            \end{itemize}
        \end{itemize}
    \end{block}
    \vspace{1em}
    \emph{„Datamining to the rescue“}
\end{frame}
\section{Datamining}

similarity measures
dissimilarity measures
supervised
unsupervised


\subsection{Knowledge Discovery in Databases (KDD)}

\subsection{Support Vector Machines}

\subsection{K-Means Clustering}

\subsection{Canopy Clustering}

\subsection{K-Nearest-Neighbor (KNN)}

\subsection{Entscheidungsbäume}
\section{Real World Mobile Sensor Data Mining}

\subsection{Ein Markt Entsteht}

\begin{frame}
    \frametitle{\insertsubsection \ -- \cite{lockhart2011}}
    \begin{itemize}
        \setlength\itemsep{1em}
        \item Menschliches Verhalten in Daten messbar? -- Universitärer Interessenanstieg an dem Thema
        \item Forschungsprojekte wie zB. „WISDM Lab“ Wireless Sensor Data Mining Laboraty entstehen\footnote{Fordham University, Bronx, NY \url{http://www.cis.fordham.edu/wisdm/}}
        \vspace{0.7em}
        \item Mehr Wissen über Nutzer -- monetäre Interessen wachsen
        \item Neue Möglichkeiten, Produkte \& Anwendungen zu verbessern
    \end{itemize}
\end{frame}

\subsection{iOS Health}

\begin{frame}
    \frametitle{\insertsubsection}
    \center
    \Large
    \emph{„The all-new Health app has been redesigned to make it easier to learn about your health and start reaching your goals. \textcolor{blue}{It consolidates health data from iPhone, Apple Watch, and third-party apps you already use}, so you can view all your progress in one convenient place. \textcolor{blue}{And it recommends other helpful apps} to round out your collection — making it simpler than ever to move your health forward.”} \cite{iosHealth}
\end{frame}

\begin{frame}
    \frametitle{\insertsubsection}
    Aus dem iOS Security Whitepaper \cite{iosSecurity} -- HealthKit aggregiert...
    \vspace{.8em}
    \begin{itemize}
        \setlength\itemsep{0.6em}
        \item „Health Data“: \emph{Größe, Gewicht, gegangene Distanz, Blutdruck uvm.}
        \item „Management Data“: \emph{Zugriffsberechtigungen, Angeschlossene Geräte, „Scheduling Daten“ wenn Apps gestartet werden uvm.}
    \end{itemize}
    \vspace{.8em}
    Aus der Apple HealthKit API Reference \cite{hkApi}:
    \vspace{.8em}
    \begin{itemize}
        \setlength\itemsep{0.6em}
        \item „Store” in der Cloud
        \item Gerät überträgt (Sensor-) Daten
    \end{itemize}
\end{frame}

\begin{frame}
    \frametitle{\insertsubsection}
    Brainstorming: was passiert bei HealthKit alles?\\
    \vspace{.6em}
    „Context“ umfasst...
    \vspace{.6em}
    \begin{itemize}[<+->]
        \setlength\itemsep{0.5em}
        \item Biologische Nutzer Daten
        \item Perönliche Nutzer Daten
        \item Gerät der Anwendungsausführung (Phone, Uhr etc.)
        \item Daten anderer Apps auf dem Gerät
        \item Geografische Bewegungsdaten des Geräts / Nutzers
        \item ...
    \end{itemize}
    \pause
    \vspace{.6em}
    Außerdem wird die Nutzbarmachung dieses Contexts als API Feature für Developer emöglicht.\\
    \pause
    \vspace{.6em}
    \emph{Ist das eine „biologische Query-API“ auf einem Menschen?}
\end{frame}
\chapter{Zusammenfassung}

Dieser Aufsatz hat zunächst den Begriff von „Kontext“ geprägt und eine anschauliche Motivation geliefert für die Nutzung von Kontextwissen. Es wurde ein Überblick über Datamining im Allgemeinen geboten. Anschließend wurden Datamining-Techniken und Algorithmen in Zusammenhang gestellt mit der Nutzbarmachung von mobilen (Sensor-) Daten.

Am Beispiel von iOS Health wurde herausgearbeitet, welche großartigen Möglichkeiten entstehen können, wenn Datamining im mobilen Bereich eingesetzt wird. Zugleich wurde kritisch darauf hingewiesen, wie viele private und persönliche Daten von Menschen erfasst werden.

Abschließend lässt sich festhalten, dass mit Datamining und der Analyse von (Sensor-) Daten großartige Möglichkeiten zur Umsetzung von Nutzerfeatures entstehen. Dabei ist es meiner Meinung jedoch sehr wichtig, sorgsam mit der Datenfülle umzugehen und die Privatsphäre der Nutzer vor wirtschaftliche Interessen zu stellen.

\cleardoublepage

% VERZEICHNISSE (Abbildungen, Tabellen)
% Literatur 
%\bibliographystyle{alphadin}
\bibliographystyle{apalike}
\bibliography{../bib}
\cleardoublepage

\end{document}