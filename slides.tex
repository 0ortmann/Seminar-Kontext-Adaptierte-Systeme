\documentclass{beamer}
%\documentclass[handout]{beamer}

\usepackage[ngerman]{babel}
\usepackage[utf8]{inputenc}
\usepackage[T1]{fontenc}
\usepackage{lmodern}

\usetheme{vsis}

% Vor jedem neuen Teil (\part) eine Folie mit Titel anzeigen
\AtBeginPart{\frame{\partpage}}

% Vor jedem neuen Abschnitt (\section) eine Folie mit Gliederung für diesen Abschnitt zeigen
\AtBeginSection{\frame{\frametitle{\insertsection}\tableofcontents[currentsection,hideothersubsections]}}

% für Grafik-Beispiel:
\usepackage{tikz}
\usetikzlibrary{arrows,positioning,fit,backgrounds,shapes}

% Daten dieser Pr?entation
\title{Mobile kontextadaptive Dienste und Systeme}
\subtitle{Data Mining \& Analyse von Sensordaten }
\author{Felix Ortmann}
\date{\today}
\institute[Uni-HH]{Universität Hamburg \\ Fakultät für Mathematik, Informatik und Naturwissenschaften \\ Department Informatik \\ Zentrum für Verteilte Informations- und Kommunikationssysteme \\ Arbeitsbereich Verteilte Systeme und Informationssysteme}


\begin{document}

\maketitle

\begin{frame}{Gliederung}
  \tableofcontents[hideallsubsections]
\end{frame}


\section{Einführung}


\subsection{Motivation}

\begin{frame}
    \frametitle{\insertsubsection} 
    \begin{itemize}
        \setlength\itemsep{1em}
        \item Durch mobile Geräte werden zunehmend mehr Daten erhoben
        \item „Context“ Wissen
        \item Unterschiedlichste Nutzungsszenarien
        \item Nutzbar für wen, wozu?
        \item Besondere Verfahren notwendig, um nutzbare Informationen aus Datenflut zu extrahieren
    \end{itemize}
\end{frame}


\subsection{Zielsetzung}

\begin{frame}
    \frametitle{\insertsubsection} 
    \begin{itemize}
        \setlength\itemsep{1em}
        \item Datamining und Machinelearning mit/auf mobilen Geräten
        \item Hearausstellung diverser Problematiken bezogen auf den mobilen Bereich
        \item Datamining, Techniken und Algorithmen
        \item Real-World Beispiele
    \end{itemize}
\end{frame}

\section{Sensoren}

\subsection{Context Awareness}

\subsection{Vorverarbeitung}

\subsection{Mobilgeräte}

\subsection{Haushaltsgeräte}

\subsection{Autos}
\section{Datamining}

similarity measures
dissimilarity measures
supervised
unsupervised


\subsection{Knowledge Discovery in Databases (KDD)}

\subsection{Support Vector Machines}

\subsection{K-Means Clustering}

\subsection{Canopy Clustering}

\subsection{K-Nearest-Neighbor (KNN)}

\subsection{Entscheidungsbäume}
\section{Anwendungen}

\subsection{Wo wird mining mit mobile daten gemacht?}
\chapter{Zusammenfassung}

Dieser Aufsatz hat zunächst den Begriff von „Kontext“ geprägt und eine anschauliche Motivation geliefert für die Nutzung von Kontextwissen. Es wurde ein Überblick über Datamining im Allgemeinen geboten. Anschließend wurden Datamining-Techniken und Algorithmen in Zusammenhang gestellt mit der Nutzbarmachung von mobilen (Sensor-) Daten.

Am Beispiel von iOS Health wurde herausgearbeitet, welche großartigen Möglichkeiten entstehen können, wenn Datamining im mobilen Bereich eingesetzt wird. Zugleich wurde kritisch darauf hingewiesen, wie viele private und persönliche Daten von Menschen erfasst werden.

Abschließend lässt sich festhalten, dass mit Datamining und der Analyse von (Sensor-) Daten großartige Möglichkeiten zur Umsetzung von Nutzerfeatures entstehen. Dabei ist es meiner Meinung jedoch sehr wichtig, sorgsam mit der Datenfülle umzugehen und die Privatsphäre der Nutzer vor wirtschaftliche Interessen zu stellen.


\section{Literatur}

\nocite{*}

\begin{frame}[allowframebreaks]{\insertsubsection}
	\begingroup
	\small
	\beamertemplatebookbibitems
	\bibliographystyle{plain}
	\bibliography{bib}
	\endgroup
\end{frame}

\end{document}
